\documentclass[12pt,a4paper]{article}
\usepackage[margin=1in]{geometry}
\usepackage{tikz}
\usepackage{amsmath}
\usepackage{enumerate}
\usepackage{enumitem}
\usepackage{amssymb}
\usepackage{amsfonts}
\usepackage{amsthm}
\usepackage[english]{babel}
\usepackage{multicol}
\usepackage{graphicx}
\usepackage{esint}
\usepackage[utf8]{inputenc}

\newcommand{\problem}[2]{\noindent \\ \textbf{#1} #2 \\ \hspace{0.75em} \noindent \\}
\newcommand{\pnum}[1]{\item[\textbf{#1}]}
\newcommand{\Arg}{\,\text{Arg}\,}
\newcommand{\Log}{\,\text{Log}\,}
\newcommand{\Arctan}{\text{Arctan }}
\newcommand{\N}{\mathbb{N}}
\newcommand{\Z}{\mathbb{Z}}
\newcommand{\Q}{\mathbb{Q}}
\newcommand{\R}{\mathbb{R}}
\newcommand{\C}{\mathbb{C}}

\renewcommand{\Im}{\text{Im}\,}
\renewcommand{\Re}{\text{Re}\,}

\title{Complex Analysis Summer Work}
\author{Sameed Pervaiz}
\date{\today}

\begin{document}

\maketitle

\section*{Chapter I: The Complex Number System}

\begin{enumerate}[label={\bfseries I.4.\arabic*}]
	\item If $z=1+2i$ and $w=3+4i$, express the following in the form $x+iy$:
	\begin{enumerate}
		\item $3z+iw = 3(1+2i) + i(3+4i) = -1 + 9i$ \\
		\item $2z^2-z\overline{w} = 2(1+2i)^2 - (1+2i)(3-4i) = -17 + 6i$ \\
	\end{enumerate}
	
	The rest of this is kinda useless so...

	% TODO: Fix indentation.

	\item Verify that $\overline{z+w} = \overline{z} + \overline{w}$, $\overline{zw} =
	\overline{z}\overline{w}$, and $|zw|=|z||w|$ for all complex numbers $z$ and $w$. Assuming that
	$w \neq 0$, show also that $\overline{z/w}=\overline{z}/\overline{w}$ and $|z/w|=|z|/|w|$.

	These are also trivial to prove...

	\item Confirm that the identity $1+z+\cdots+z^n = (1-z^{n+1})/(1-z)$ holds for every non-negative
	integer $n$ and every complex number $z$, save for $z=1$.

	\begin{proof}
		Take $z$ to be a complex number. Then, we prove the base case, $n=0:$
		\begin{align}
			1 &= (1-z)/(1-z) \\
			1 &= 1
		\end{align}
		Now, assume the case $n=n$, and prove the $n+1$ case:
		\begin{align}
			1 + z + \cdots + z^n &= (1-z^{n+1})/(1-z) \\
			1 + z + \cdots + z^{n-1} &= (1-z^{n+1} + z^{n+1}(1-z))/(1-z) \\
			&= (1-z^{(n+1)+1})/(1-z)
		\end{align}
	\end{proof}

	\item Establish the parallelogram law for complex numbers $z,w$: $|z+w|^2 + |z-w|^2 = 2|z|^2 + 2|w|^2$
	\begin{align}
    |z+w|^2 + |z-w|^2 &= (a+c)^2 + (b+d)^2 + (a-c)^2 + (b-d)^2 \\
    &= a^2 + 2ac + c^2 + b^2 + 2bd + d^2 + a^2 -2ac + c^2 + b^2 -2bd + d^2 \\
    &= 2(a^2 + c^2) +2(b^2 + d^2) \\
    &= 2|z|^2 + 2|w|^2
	\end{align}

	\item Given nonzero $w,z$, prove that $|z+w|=|z|+|w|$ is true iff $|w=tz|$ for some positive $t$.
	\begin{align}
    \sqrt{(a+c)^2+(b+d)^2} &= \sqrt{a^2+b^2} + \sqrt{c^2 + d^2} \\
    (a+c)^2 + (b+d)^2 &= a^2 + b^2 + c^2 + d^2 + 2\sqrt{(a^2+b^2)(c^2+d^2)} \\
    ac + bd &= \sqrt{a^2c^2 +b^2c^2 + a^2d^2 + b^2d^2} \\
    a^2c^2 + b^2d^2 + 2acbd &= a^2c^2 + b^2c^2 + a^2d^2 + b^2d^2 \\
    2adbc &= b^2c^2 + a^2d^2 \\
    0 &= a^2d^2 - 2adbc + b^2c^2 \\
    &= (ad-bc)^2 \\
    ad &= bc \\
    \frac{a}{c} &= \frac{b}{d} = t \\
    \implies a = ct &\land b=dt \\
    \implies a+bi &= t(c+di) \\
    \implies z &= t(w)
	\end{align}
\end{enumerate}

\section*{Chapter 2: Rudiments of Plane Topology}

\begin{enumerate}[label={\bfseries II.5.\arabic*}]
	\item  Verify that $\Delta(z_0, r)$ is an open set and that $\overline{\Delta}(z_0,r)$ is a
	closed set.
	
	\begin{proof}
		A given point $z$ in $\Delta \equiv \Delta(z_0, r)$ exists on the radius of the open disk. Thus,
		we can explicitly construct an open disk $\Delta(z, r-|z_0-r|)$ which lies completely in $\Delta$.

		For any closed disk $\overline{\Delta} \equiv \overline{\Delta}(z_0, r)$, we have that its complement
		$\C - \overline{\Delta}$ is the plane minus the closed disk. Thus any point $z$ in the complement
		exists on a line segment of radius $|z-z_0|$ connecting $z$ and $z_0$. We can then similarly
		construct an open disk of radius $|z-z_0|-r$ centered on $z$,	and thus $\overline\Delta$ is closed.
	\end{proof}

	\item Prove Theorems 1.1 and 1.2.
		\begin{enumerate}[label={\bfseries Theorem 1.\arabic*}]
			\item The collection of open subsets of $\C$ is closed under unions and finite intersections.
			\begin{proof}
				Given $V \equiv \cup U_i$, take any point $p \in V$. It follows that $p$ is an element of at
				least one $U_i$, which is an open set. There then exists an open disk centered on $p$ in $U_i$,
				and therefore in $V$ as well. Thus $V$ is open.

				Now, take $V \equiv \cap U_i$, where the index set is finite. Provided the intersection is
				nonempty, a given point $p \in V$ must be an element of every $U_i$. Hence there is a set of
				open disks $\Delta_i$ each centered on $p$, and contained fully in $U_i$. Take the minimum
				radius min$(\{r_i\})$. Then the open disk of this radius is contained in each $U_i$, and as
				such, $V$. Thus $V$ is open.
			\end{proof}

			\item The collection of closed subsets of $\C$ is closed under intersections and finite unions.
			\begin{proof}
				De Morgan's laws allow us to equate the arbitrary intersection of closed sets in $\C$, and
				equate it to the arbitrary union of each set's complements, which we know to be open. Taking
				another complement, it follows that closed sets are closed under arbitrary intersections.

				On the other hand, suppose we have a finite union of closed sets. By the same argument, the
				complement is a finite intersection of open sets. Since this is open, the complement is closed,
				and so closed subsets are closed under finite unions.
			\end{proof}
		\end{enumerate}

	\item Classify each of the following sets as open, closed, or neither open nor closed:
	(\textit{Hint}. At least for (i)-(v) it may help to view hte set graphically.)

	\begin{enumerate}[label=(\roman*)]
		\item $A = \{z : -\pi < \Im z \leq \pi \}$
			\begin{enumerate}
				\item[] Half-open infinite strip between $y=-\pi$ and $y=\pi$: neither.
			\end{enumerate}
		\item $B = \{ z : 1 < |z| < 2\}$
			\begin{enumerate}
				\item[] Torus of inner radius 1, outer radius 2: open.
			\end{enumerate}
		\item $C = \{ z : |\Re z| + |\Im z| \leq 1\}$
			\begin{enumerate}
				\item[] Unit disk under the 1-norm: closed
			\end{enumerate}
		\item $D = \{ : 0 < \max\{x,y\} \leq 1\}$
			\begin{enumerate}
				\item[] Punctured unit disk under the taxicab metric: neither.
			\end{enumerate}
		\item $E = \{z : y > x^2 \}$
			\begin{enumerate}
				\item[] Subset of the plane above the parabola $y=x^2$: open.
			\end{enumerate}
		\item $F = \{z : \Re z \text{ and } \Im z \text{ are rational}\}$
			\begin{enumerate}
				\item[] Neither. No open ball contains only rationals, and rationals can have
				irrational limits.
			\end{enumerate}
	\end{enumerate}

	\item Determine the boundary, closure, and interior of each of the sets listed in Exercise 5.3

	\begin{enumerate}[label=(\roman*)]
		\item $A = \{z : -\pi < \Im z \leq \pi \}$
			\begin{enumerate}
				\item[] Boundary: horizontal lines at $\pm\pi$
				\item[] Closure: the closed horizontal strip.
				\item[] Interior: the fully open horizontal strip.
			\end{enumerate}
		\item $B = \{ z : 1 < |z| < 2\}$
			\begin{enumerate}
				\item[] Boundary: Circles of radius 1 and 2 centered on 0.
				\item[] Closure: Closed torus.
				\item[] Interior: Same as the set.
			\end{enumerate}
		\item $C = \{ z : |\Re z| + |\Im z| \leq 1\}$
			\begin{enumerate}
				\item[] Boundary: Symmetric reflection of $y=1-x$ in each quadrant.
				\item[] Closure: Same as set.
				\item[] Interior: Same as set, minus boundary.
			\end{enumerate}
		\item $D = \{ : 0 < \max\{x,y\} \leq 1\}$
			\begin{enumerate}
				\item[] Boundary: Origin plus unit square.
				\item[] Closure: Closed unit square.
				\item[] Interior: Same as the set, minus unit square.
			\end{enumerate}
		\item $E = \{z : y > x^2 \}$
			\begin{enumerate}
				\item[] Boundary: Parabola $y=x^2$.
				\item[] Closure: Subset of the plane including parabola and above..
				\item[] Interior: Same as the set.
			\end{enumerate}
		\item $F = \{z : \Re z \text{ and } \Im z \text{ are rational}\}$
			\begin{enumerate}
				\item[] Boundary: The plane.
				\item[] Closure: The plane.
				\item[] Interior: Empty.
			\end{enumerate}
	\end{enumerate}

	\item Prove Theorem 1.3.
		\begin{enumerate}
			\item[\textbf{Theorem 1.3.}] \emph{The boundary $\partial A$ and closure $\overline A$ of a
			subset $A$ of $\C$ are closed sets}.
			\begin{proof}
				A point $z$ in the boundary $\partial A$ is such that, for any positive $r$, $\Delta(z, r)$
				has non-empty intersection with both $A$ and $\C - A$. It follows, then, that any point $z'$
				in the complement $\C - \partial A$ is such that, for \emph{some} $r > 0$, $\Delta(z', r)$
				has empty intersection with either $A$ or $\C - A$. Since, for any $z_0$ in	the disk about $z'$,
				the smaller disk $\Delta(z_0, r-|z'-z_0|)$ is a subset of $\Delta(z',r)$, it follows that every
				element of $\Delta(z', r)$ is an element of $\C - \partial A$, and so any arbitrary point in
				the complement is an interior point. Hence the complement is open, and $\partial A$ is closed.

				Suppose a point $z$ is in the closure $\overline A = A \cup \partial A$. Then, for every $r > 0$,
				$A \cap \Delta(z, r)$ is non-empty. But then, any point $z'$ in the complement $\C-\overline A$
				is such that, for some $r > 0$, $A \cap \Delta(z', r)$ is empty. Thus, for any such point $z'$,
				the disk $\Delta(z',r)$ exists entirely in $\C - \partial A$, and so $\C - \partial A$ is open.
				Taking a complement again, it follows that $\partial A$ is closed.
			\end{proof}
	\end{enumerate}

	\item Compute the limit of $(z_n)$ when
	\begin{enumerate}[label=(\roman*)]
		\item $z_n = i^{n!} + 2^{-n}$
			\begin{itemize}
				\item[] Divergent, since $2^{-n}$ converges to 0, but the $i$ term oscillates.
			\end{itemize}
		\item $z_n = (in^3 + 1)/(2n^3+n^2)$
			\begin{itemize}
				\item[] Converges to $\frac{i}{2}$.
			\end{itemize}
		\item $z_n = 2^{-n + i\sqrt{n}}$
			\begin{itemize}
				\item[] Converges to 0.
			\end{itemize}
		\item $z_n = \sqrt[n]{z}$
			\begin{itemize}
				\item[] Converges to 0.
			\end{itemize}
		\item $z_n = \sqrt[n]{n}$
			\begin{itemize}
				\item[] Take the log, and then exponentiate. $\lim z_n = \lim \exp [(\log n)/n]=1$. 
			\end{itemize}
		\item $z_n = (1+n^{-1})^n + i(1-n^{-1})^n$
			\begin{itemize}
				\item[] Converges to $e + i\frac{1}{e}$.
			\end{itemize}
		\item $z_n = z^n/n!$
			\begin{itemize}
				\item[] Converges to 0.
			\end{itemize}
		\item $z_n = n\sin(1/n) + ine^{-n}$
			\begin{itemize}
				\item[] Converges to 0.
			\end{itemize}
	\end{enumerate}

\item Assuming the usual algebraic rules for real sequential limits, derive Theorem 1.5 from Lemma 1.4.
\end{enumerate}
\end{document}
