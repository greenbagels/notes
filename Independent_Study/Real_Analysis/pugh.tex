\documentclass[11pt, a4paper, latinreim, shortsets]{notes}

\newcommand{\topo}{\mathcal{T}}
\newcommand{\cl}[1]{\overline{#1}}

\title{Real Analysis Summer Work \\
\large Exercises from Pugh's \emph{Real Mathematical Analysis (2002)}}

\author{Sameed Pervaiz}
\date{\today}

\begin{document}
\maketitle

\setcounter{chapter}{1}
\chapter{A Taste of Topology}
\begin{enumerate}[label={\bfseries 2.\arabic*}]
	\item Prove that $(0,1)$ is an open subset of $\R$ but not of $\R^2$.
	\begin{proof}
		$(0,1) \equiv B(0.5, 0.5)$ in $\R$, ergo an open set. However, in $\R^2$,
		every open ball centered on some point in $(0,1)$ contains points with nonzero $y$
		coordinate, and so the interval is not open.
	\end{proof}

	\item For which intervals $[a,b]$ in $\R$ is the intersection $[a,b] \cap \Q$ a clopen subset
	of $\Q$?
	\begin{proof}
		Since $\Q$ is a metric subspace of $\R$, it follows that all the closed sets in $\Q$
		are intersections of closed sets in $\R$, and similar for open sets. Thus the clopen sets are
		sets in which $[a,b] \cap \Q = (a,b) \cap \Q$, which happens only when $a,b$ are irrational.
		Thus $[a,b]$ with $a,b$ irrational are those such intervals.
	\end{proof}

	\item Prove directly from the definition of closed set that each single point is a closed
	subset of a metric space. Why does this imply that a finite set of points is also a closed set?
	\begin{proof}
		Let $S=\{x\}$ be a singleton set. Then, only one sequence of points $(x_n)=(x,x,x,x,\cdots)$
		exists in $S$, with limit $x$, which is contained in the set. Thus $S$ is closed. Since a
		finite union of closed sets is closed, it follows that any sets containing finitely many
		points are closed.
	\end{proof}

	\item Prove that $S$ clusters at $p$ if and only if, for each $r > 0$, there is a point
	$q \in M_{r}(p) \cap S$ such that $q \neq p$.
	\begin{proof}
	Suppose $S$ clusters at $p$. Then, every neighborhood of $p$ contains infinitely many points of $S$.
	If a set contains infinitely many points, then it contains at least two distinct points. Since $p$
	is one of those points, it follows that there exists a distinct point $q\in M_r(p)\cap S$ in every
	neighborhood of $p$.

	Now suppose that every neighborhood of $p$ contains another distinct point $q$ in $S$. Let
	$r' = d(p,q)$. Then, it follows that the neighborhood $M_{r'}(p) \cap S$ contains a point distinct
	from $p$ and $q$. Thus, for any finite set of points, we can find another distinct point closer to
	$p$, and so the set must be infinite.
	\end{proof}

	\item Prove that a set $U \subset M$ is open if and only if none of its points are limits of
	its complement.
	\begin{proof}
	Suppose $U$ is open. Then, $U^c$ is closed and every one of its limits is contained in $U^c$.
	Since no point in $M$ is in both $U$ and $U^c$, none of $U^c$'s limits lie in $U$.

	Now, suppose none of $U$'s points are limits of its complement. Then, all of the limits of
	$U^c$ must lie in $U^c$, and $U^c$ is closed. Since the complement of a closed set is open,
	$U$ is open.
	\end{proof}

	\item If $S,T \subset M$, a metric space, and $S \subset T$, prove that
	\begin{enumerate}[label=\alph*)]
		\item $\overline{S} \subset \overline{T}$
	\begin{proof}
		Since $S \subset T$, every limit of points in $S$ is a limit of points in $T$. Since closure
		is the union of a set and its limit points, it follows that $\overline{S} \subset \overline{T}$
	\end{proof}
		\item int$(S)\subset$int$(T)$
	\begin{proof}
		If a point $p$ is in the interior of $S$, then there exists a neighborhood in $S$ that contains
		$p$. Since $S \subset T$, every neighborhood in $S$ is also a neighborhood in $T$, and so $p$
		is in the interior of $T$. Thus int$(S)\subset$int$(T)$.
	\end{proof}
	\end{enumerate}

	\item Construct a set with exactly three cluster points.
	\begin{proof}
		The set $\{\frac{1}{n} : n \in \N\}$ has a single limit point of 0. If we include three shifted
		copies of this set, then, it follows that there are three cluster points.
	\end{proof}

	\item If $A \subset B \subset C$, $A$ is dense in $B$, $B$ is dense in $C$, prove that
	$A$ is dense in $C$.
	\begin{proof}
		$A$ dense in $B$ means $B \subset \cl{A}$, and $B$ dense in $C$ means $C \subset \cl{B}$.
		Since closure is idempotent, it follows that $\cl{B} \subset \cl{A}$. Thus $C \subset \cl{A}$,
		and so $A$ is dense in $C$.
	\end{proof}

\item Is the set of dyadic rationals (the denominators are powers of 2) dense in $\Q$?
	In $\R$? Does one answer imply the other? (Recall that $A$ is dense in $B$ iff $A \subset B$ and
	$\cl{A} \supset B$.)

	\begin{enumerate}[label=\alph*)]
		\item Find a metric space in which the boundary of $M_rp$ is not equal to the sphere of radius
		$r$ at $p$, $\{x\in M:d(x,p) = r\}$.
		\begin{itemize}
			\item[] Any metric space with the discrete metric has all sets open, and so the union of
			open sets contained in a set $A$ is just the union of subsets of $A$, which means
			$\text{int}(A)=A$. Thus $\partial A = \varnothing$ for every set $A$.
		\end{itemize}
		\item Need the boundary be contained in the sphere?
		\begin{itemize}
			\item[] Yes, for if there were a boundary point $x$ outside of the n-sphere, then there
			would exist an $\epsilon$-neighborhood of the point fully outside of $M_rp$, thus making
			$x$ not a boundary point.
		\end{itemize}
	\end{enumerate}

	\item Let $\topo$ be the collection of open subsets of a metric space $M$, and $\mathcal{K}$
	the collection of closed subsets. Show that there is a bijection from $\topo$ onto $\mathcal{K}$.
	\begin{proof}
		We approach the problem directly by constructing an explicit bijection $f$
		mapping every open subset $U$ to its complement $M - U$. Suppose that $f(U) = f(V)$
		for some two open sets $U,V$. Then, $M - U = M - V$. Taking complements again,
		it follows $U = V$. Thus $f$ is injective.

		Now, for any closed set $V$ in $M$, there exists an open set $M - V$ such that
		$f(M-V) = M-(M-V) = V$. Thus $f$ is surjective as well, and so bijective, too.
	\end{proof}

	\item Let $M$ be a metric space with the discrete metric, or more generally a
	homeomorph of $M$.

	\begin{enumerate}[label=\alph*)]
	\item Prove that every subset of $M$ is clopen
	\begin{proof}
		Consider an open ball of radius $\epsilon < 1$ centered on some point $x \in M$.
		Since $d(x,y) = 1$ for all $y\neq x$, it follows that $M_\epsilon(x) = x$.
			Thus, every set that contains a point $x$ also contains an $\epsilon$-neighborhood of $x$,
			and so every set is open.

			Now consider any convergent sequence of points $(x_n)$ in a subset of $M$.
			If a limiting point $p$ exists, it follows that, for any $\epsilon > 0$, there exists
			$N > 0$ such that $d(x_n,p) < \epsilon$ for all $n > N$. But for any $\epsilon < 1$,
			$d(x_n, p)< \epsilon$ implies that $x_n = p$.

			Thus, any convergent sequence in a clopen metric space is simply an eventually-constant
			sequence. From this, it is clear that any convergent sequence of points in a set $U$
			has a limit in $U$, and so all sets are closed.
	\end{proof}
	\item Prove that every function defined on $M$ is continuous.
			\begin{proof}
				Since all sets are open (equiv. closed), it follows that any mapping between sets must
				map open (equiv. closed) sets to open (equiv. closed) sets, and is thus continuous.
			\end{proof}
	\item Which sequences converge in $M$?
			\item[] Any eventually-constant sequence, as proved above.
	\end{enumerate}

	\item Show that every subset of $\N$ is clopen. What does this tell you about every
	function $f : \N \rightarrow M$, where $M$ is a metric space?

	\begin{proof}
		The proof of clopenness is very similar to that of a metric space endowed with the discrete
		metric, as $\N$ is a discrete space (all points are isolated), so we won't rewrite it.
		We can also note that every function on $\N$ is continuous.
	\end{proof}

	\item from a point $p$ in a metric space $M$ to a non-empty subset
	$S \subset M$ is defined to be $\text{dist}(p,S) = \text{inf}\{d(p,s) : s \in S\}$

	\begin{enumerate}[label=\alph*)]
		\item Show that $p$ is a limit of $S$ iff dist$(p,S)=0$.
			\begin{proof}
				Suppose $p$ is a limit of some sequence $(x_n)$ of points in $S$. Then, for every
				$\epsilon > 0$, there exists $N > 0$ such that $d(x_n, p) < \epsilon$ for all $n > N$.
				Since $d$ is non-negative, 0 is a lower bound for $d(x_n, p)$. Since we can always
				find an $x_m$ such that $d(x_m, p) < d(x_n, p) < \epsilon$ for any positive $\epsilon$,
				it follows that 0 is the greatest lower bound.

				Now, suppose the opposite --- that 0 is the greatest lower bound of distances between
				$p$ and points in $S$. Pick a point $x_1$ such that $d(x_1, p) < 1$. Then, pick another
				point $x_2$ such that $d(x_2, p) < \frac{1}{2}$. In this way, we can construct a
				sequence of points $(x_n)$ such that $d(x_n, p) < \frac{1}{n}$ for any $N$. Given
				any $\epsilon > 0$, then, we can find an $N = \lceil\frac{1}{\epsilon}\rceil$ such that
				$d(x_n, p) < \epsilon$ for all $n > N$, and the converse holds.
			\end{proof}
		\item Show that $p\mapsto \text{dist}(p,S)$ is a uniformly continuous function of $p \in M$.
			\begin{proof}
				Define a function $f : M \rightarrow \R$ such that $f(p) = $dist$(p,S)$. Uniform
				continuity requires that, for every $\epsilon > 0$, there exists $\delta > 0$ such
				that $d(p,q) < \delta$ implies $d(f(p),f(q)) < \epsilon$. 

				Suppose $d(f(p),f(q)) < \epsilon$. It follows that
				\begin{align}
					|dist(p,S) - dist(q,S)| &= |\text{inf}(\{d(p, s_n)\} - \text{inf}(\{d(q, s_n)\}| \\
											&< \epsilon 
				\end{align}
				But the infimum of a difference is less than the difference of the infima, so
			\end{proof}
	\end{enumerate}

	\item What is the set of points in $\R^3$ at distance $1/2$ from the unit circle in the plane?
	\begin{itemize}
		\item[] It's a torus with inner radius 0, and outer radius 1.
	\end{itemize}

	\item Show that $S \subset M$ is somewhere dense in $M$ iff int$(\overline{S}) \neq \varnothing$.
	That is, $S$ is nowhere dense in $M$ if and only if its closure has empty interior.

	\begin{proof}
		Suppose the closure of $S$ has empty interior. Since the interior of a set is the union of all
		open subsets of the set, it follows that the empty set is the only open set contained in the closure
		of $S$, and thus $S$ itself. Thus the intersection with any open set $U \subset M$ is empty, so
		there exists no open non-empty sets $U$ such that $\overline{S \cap U} \supset U$.
	\end{proof}

	\item Assume that $f : M \to N$ is a function from one metric space to another which satisfies
	the following condition: if a sequence $(p_n)$ in $M$ converges then the sequence $(f(p_n))$ in $N$
	converges. Prove that $f$ is continuous.

	\begin{proof}
		If, for every $\epsilon > 0$ there exists $N > 0$ such that for all $n > N$, $d(p_n, p) < \epsilon$,
		then for every positive $\epsilon$ there exists $M$ such that $m > M$ implies $d(f(p_n),
		f(p)) < \epsilon$.
	\end{proof}

	\item The simplest type of mapping from one metric space to another is an \textbf{isometry}.
	It is a bijection $f : M \to N$ that preserves distance in the sense that for all $p, q \in M$,
	\begin{align}
		d_{N}(fp, fq) = d_M (p,q)
	\end{align}
	If there exists an isometry from $M$ to $N$ then $M$ and $N$ are said to be isometric, $M \equiv N$.
	You might have two copies of a unit equilateral triangle, one centered at the origin and one centered
	elsewhere. They are isometric. Isometric metric spaces are indistinguishable as metric space.

	\begin{enumerate}[label=\alph*)]
		\item Prove that every isometry is continuous.
			\begin{proof}
				Since $d(fp, fq) = d(p, q)$, it follows $\delta = \epsilon$ guarantees continuity for any
				$p,q \in N$. Indeed, isometries are even uniformly continuous!
			\end{proof}
		\item Prove that every isometry is a homeomorphism.
			\begin{proof}
				Recall that homeomorphisms are continuous bijections with continuous inverses. Since $f$ is a
				continuous bijection, we just have to prove the inverse is continuous. Since $(f^{-1}fp,f^{-1}fq)
				= (p,q)$, it follows $d(f^{-1}fp, f^{-1}fq) = d(p,q) = d(fp, fq) = d(f^{-1}fp,d^{-1}fq)$.
				This means the inverse is also an isometry, and is thus continuous, and every isometry is a
				homeomorphism.
			\end{proof}
		\item Prove that [0,1] is not isometric to [0,2].
			\begin{proof}
				Not sure if this is obvious or not; are we assuming a choice of metric?
			\end{proof}
	\end{enumerate}

	\item Prove that isometry is an equivalence relation: if $M$ is isometric to $N$, show that $N$
	is isometric to $M$; show that any $M$ is isometric to itself (what mapping of $M$ to $M$ is an
	isometry?); if $M$ is isometric to $N$ and $N$ is isometric to $P$, show that $M$ is isometric to $P$.

	\begin{proof}
		Easy. We showed that the inverse of an isometry is an isometry, thus $N$ is isometric to $M$.
		The identity map is trivially an isometry, so every $M$ is isometric to itself. Lastly, since
		the composition of bijections is bijective, given two isometries $f : M \to N$, $g : N \to P$,
		it follows that
		\begin{align}
			d(fp, fq) &= d(p,q) \\
			d(gfp, gfq) &= d(fp, fq) \\
									&= d(p,q)
		\end{align}
		and so the composition of $f$ and $g$ is also an isometry.
	\end{proof}

	\item Is the perimeter of the square isometric to the circle? Homeomorphic?

	\item Which capital letters of the Roman alphabet are homeomorphic? Are any isometric?
	Explain.

	\item Is $\R$ homeomorphic to $\Q$? Explain.

	No; $\Q$ is disconnected, while $\R$ is connected, and so they cannot be homeomorphic.

	\item Is $\Q$ homeomorphic to $\N$? Explain.

	No, if a continuous bijection from $\Q$ to $\N$ existed, then every open set in $\N$ has an open
	preimage, and every closed set in $\N$ has a closed preimage. But since every set in $\N$ is clopen,
	the preimage of every set would have to be clopen. So take a singleton set in $\N$. Under a bijective
	map to $\Q$, it has a singleton preimage. But no singleton in $\Q$ is open, and so no homeo can exist.

	\item An ant walks on the floor, walls, and ceiling of a cubical room. What metric is
	natural	for the ant's view of the world? What metric would a spider consider natural? If the ant wants
	to walk	from a point $p$ to a point $q$, how could it determine the shortest path?

	\item Assume that $N$ is an open metric subspace of $M$ and that $U \subset N$.
	\begin{enumerate}[label=\alph*)]
		\item Prove that $U$ is open in $N$ if and only if it is open in $M$.
			\begin{itemize}
				\item[] \begin{proof}
						If $U$ is open in $N$, then it is open in $M$, as every neighborhood of a point in $N$
						is a neighborhood of a point in $M$. Conversely, if $U$ is open in $M$, then there exists
						an open neighborhood around every point in $U$. Since $U$ is a subset of $N$, which is open,
						these neighborhoods can exist inside $N$, and thus $U$ is open in $N$.
					\end{proof}
			\end{itemize}
		\item Conversely, prove that if openness of $S \subset N$ is equivalent to openness in $M$, then
			$N$ is open in $M$.
			\begin{itemize}
				\item[] \begin{proof}
						If $S$ being open in $N$ is equivalent to being open in $M$, then $N$ being open in itself
						implies it is open in $M$ as well.
				\end{proof}
			\end{itemize}

		\item Do the same for closedness.
			\begin{itemize}
				\item[] \begin{proof}
						Assume $N$ is a closed metric subspace of $M$, and that $U \subset N$. Then if $U$ is closed
						in $N$, all its limits are contained in $U$, which is also contained in $M$, and so it is
						closed in $M$. Conversely, if $U$ is closed in $M$, and is a subset of $N$, then it is also
						closed in $M$, as its limits are still inside of itself. We can also similarly say that if
						closedness is equivalent in $M$ and $N$, then $N$ being closed in itself means $N$ is closed
						in $M$.
					\end{proof}
			\end{itemize}
		\item Deduce that a clopen metric subspace $N$ is the only example in which the concepts of openness
			and closedness in the subspace agree exactly with the concepts in the big space.
			\begin{itemize}
				\item[] Kinda trivial. If closedness and openness are both in agreement, then $N$ must be closed
					and open simultaneously.
			\end{itemize}
	\end{enumerate}

	\item Consider a sequence $(x_n)$ in the metric space $\R$.
	\begin{enumerate}[label=\alph*)]
		\item If $(x_n)$ converges in $\R$, prove that the sequence of absolute values $(|x_n|)$ converges
			in $\R$.
			\begin{proof}
				Follows from the reverse triangle inequality and the convergence of $(x_n)$.
			\end{proof}
		\item Prove or disprove the converse.

			It's false; take $(x_n)$ to be given by $(-1)^n \left( \frac{1}{n} + 1\right)$. The absolute 
			value sequence converges to 1, but the regular sequence alternates between $-1$ and $1$.
	\end{enumerate}

	\item Let $(A_n)$ be a nested decreasing sequence of non-empty closed sets in the metric space
	$M$.
	\begin{enumerate}[label=\alph*)]
		\item If $M$ is complete and diam$(A_n) \to 0$ as $n \to \infty$, show that $\cap A_n$ is exactly
			one point.

			\begin{proof}
			The infinite intersection of $(A_n)$ is composed of points that are in each set $A_n$. Suppose
			that more than one point exists in this intersection. Then these points exist in every set, at
			a nonzero distance apart. Thus the diameter is nonzero, as any upper bound must be nonzero, a
			contradiction.

			Now, we'll prove 
			\end{proof}
		\item To what assertions do the sets $[n,\infty)$ provide counterexamples?
	\end{enumerate}

	\item Prove that there is an embedding of the line as a closed usbset of the plane, and there is
	and embedding of the line as a bounded subset of the plane, but there is no embedding of the line as
	a closed and bounded subset of the plane.

	\begin{enumerate}[label=\alph*)]
		\item Prove that every convergent sequence is bounded. That is, if $(p_n)$ converges in the metric
			space $M$, prove that there is some neighborhood $M_r q$ containing the set $\{p_n : n \in \N\}$.
		\item Is the same true for a Cauchy sequence in an incomplete metric space?
	\end{enumerate}

	\item A sequence $(x_n)$ in $\R$ \textbf{increases} if $n < m$ implies $x_n \leq x_m$. It
	\textbf{strictly increases} if $n < m$ implies $x_n < x_m$. The same holds for decreasing. A sequence
	is \textbf{monotone} if it increases or decreases.
	\begin{enumerate}[label=\alph*)]
		\item Prove that every sequence in $\R$ which is monotone and bounded converges in $\R$.
		\item Prove that this monotone sequence condition is equivalent to the least upper bound proprerty.
	\end{enumerate}

	\item Let $(x_n)$ be a sequence in $\R$.
	\begin{enumerate}
		\item[*a$)$] Prove that $(x_n)$ has a monotone subsequence.
		\item[b$)$] How can you deduce that any bounded sequence in $\R$ has a convergent subsequence? 
	\end{enumerate}

	\item Let $p_n$ be a sequence anf $f : \N \to \N$ a bijection. The sequence $(q_k)$ for
	$k \in \N$ with $q_k = p_{f k}$ is a \textbf{rearrangement} of $p_n$.
	\begin{enumerate}[label=\alph*)]
		\item Are limits of a sequence unaffected byu rearrangement?
		\item What if $f$ is an injection?
		\item A surjection?
	\end{enumerate}

	\item If $f : A \to B$ and $g : C \to B$ such that $A \subset C$ and for each $a \in A,
	f(a) = g(a)$ then $f$ extends to $g$. Assume that $f : S \to \R$ is a uniformly continuous function
	defined on a subset $S$ of a metric space $M$.
	\begin{enumerate}[label=\alph*)]
		\item Prove that $f$ extends to a uniformly continuous function $\overline{f} : \overline{S} \to \R$
		\item Prove that $\overline{f}$ is the unique continuous function defined on $\overline{S}$
			such that $\overline{f}(x) = f(x)$ for all $x \in S$.
		\item Prove the same things when $N$ is a complete metric space and $f : S \to N$.
	\end{enumerate}

	\item A map $f : M \to N$ is open if for each open set $U \subset M$, the image set $f(U)$
	is open in $N$
	\begin{enumerate}[label=\alph*)]
		\item Does open imply continuous?
		\item Does homeo imply open?
		\item Are open continuous bijections homeos?
		\item If $f : \R \to \R$ is a continuous surjection, is it open?
		\item If $f : \R \to \R$ is a continuous open surjection, is it a homeo?
		\item What happens in $(e)$ if $\R$ is replaced by the unit circle $S$?
	\end{enumerate}

	\item Fold a piece of paper in half.
	\begin{enumerate}[label=\alph*)]
		\item Is this a continuous transformation of one rectangle into another?
		\item Is it injective?
		\item Draw an open set in the target rectangle, and find its preimage in the original rectangle.
			Is it open?
		\item What if the open set meets the crease?
		
			The baker's transformation is a similar mapping. A rectangle of dough is stretched to twice its
			length and then folded back on itswelf. Is the transformation continuous? A formula for the
			transformation in one variable is $f(x) = 1 - |1-2x|$. The $n$-th iteration of $f$ is its $n$-fold
			composition. The orbit of a point is the set of iterated images.
		\item If x is rational prove that the orbit of $x$ is a finite set.
		\item If $x$ is irrational what is the orbit?
	\end{enumerate}

	\item Rotate the unit circle $C$ by a fixed angle $\alpha$, say $R : C \to C$. (In polar coor
	dinates, the transformation $R$ sends $(1,\theta)$ to $(1,\theta + \alpha)$.)
	\begin{enumerate}[label=\alph*)]
		\item If $\alpha/\pi$ is rational, show that each orbit of $R$ is a finite set.
		\item[*b$)$] If $\alpha / \pi$ is irrational, show that each orbit is infinite and has closure $=C$.
	\end{enumerate}
\end{enumerate}

\end{document}
